\documentclass[final,xcolor={dvipsnames,svgnames,x11names,table}]{beamer}
\usetheme{RJH}
%\usetheme{Boadilla}
\usepackage[orientation=portrait,size=a0,scale=1.3]{beamerposter}
% \usepackage[orientation=portrait, size=custom, width=84.1, height=118.9, scale=1]{beamerposter}
\usepackage[absolute,overlay]{textpos}
\usepackage{pifont}
\usepackage{ulem}
\usepackage{bm}
\usepackage{siunitx}
\usepackage[export]{adjustbox}
\usepackage{gmp}
\usepackage{smartdiagram}
\usesmartdiagramlibrary{additions}
\usepackage{booktabs}
%\usepackage{enumitem}

%\DeclareGraphicsRule{.1}{mps}{*}{}

\usepackage[listings,theorems]{tcolorbox}


\usepackage{libertine}
\setlength{\TPHorizModule}{1cm}
\setlength{\TPVertModule}{1cm}

\usepackage{tikz,tikz-3dplot} %nalipour
\usetikzlibrary{shapes,arrows, decorations.pathreplacing}%, snakes} %nalipour

\def\Put(#1,#2)#3{\leavevmode\makebox(0,0){\put(#1,#2){#3}}}

% Raised Rule Command:
%  Arg 1 (Optional) - How high to raise the rule
%  Arg 2            - Thickness of the rule
\newcommand{\raisedrule}[2][0em]{\leaders\hbox{\rule[#1]{1pt}{#2}}\hspace{13.5cm}}

\newcommand{\dotrule}[1]{%
   \parbox[]{#1}{\dotfill}}

\setbeamertemplate{bibliography entry title}{}
\setbeamertemplate{bibliography entry location}{}
\setbeamertemplate{bibliography entry note}{}
\footer{}


% \usebackgroundtemplate{%%\includegraphics[width=\paperwidth]{Figures/CLIC_canvas_accelerator_small.jpg}}


\title{\Huge{Design of a drift chamber tracking system for the IDEA experiment at FCC-ee}}
\author{\vspace*{1.5cm}{\Large{\underline{N.~Alipour~Tehrani (CERN)}, B.~Hegner, F.~Grancagnolo, P.~Janot, A.~M.~Kolano,  G.~F.~Tassielli, G.~Voutsinas}\\\vspace*{1.5cm}{\Large{2018 IEEE Nuclear Science Symposium and Medical Imaging Conference}\\ \vspace*{0.8cm}\large{10 - 17 November 2018, International Convention Center Sydney, Australia}}}}
% \footer{}
\institute{CERN}
\date{}
%\footimage{}

\usepackage[backend=bibtex]{biblatex}
\addbibresource{ref.bib}
% \setbeamertemplate{bibliography item}[text]
% \renewcommand*{\bibfont}{\footnotesize}

% tcolorbox styles
\tcbset{%
    noparskip,
    colback=white, %background color of the box
    colframe=i6colorblockbg, %color of frame and title background
    coltext=black, %color of body text
    coltitle=white, %color of title text
    fonttitle=\bfseries,
    alerted/.style={coltitle=red,
                     colframe=gray!40},
    subtcolorbox/.style={coltitle=black,
                     colframe=i6colorscheme3,
                     colback=white,
                     coltitle=i6colorblockbg},
    }


\begin{document}
\begin{frame}

\begin{textblock}{6}(0.8, 12)
\includegraphics[width=\textwidth]{Figures/logo_cern.pdf}
\end{textblock}
\begin{textblock}{10}(7, 12)
\includegraphics[width=\textwidth]{Figures/FCC-logo}
\end{textblock}
\begin{textblock}{10}(71, 11)
\includegraphics[width=\textwidth]{Figures/IEEE_logo}
\end{textblock}


%%%%%%%%%%%%%%%%%%%%%%%%%%%%%%%%%%%%%
%%% Block %%%
%%%%%%%%%%%%%%%%%%%%%%%%%%%%%%%%%%%%%
\begin{textblock}{44.5}(0.5, 18)
  \begin{tcolorbox}[title=The Future Circular Collider Experiment (FCC)]

  \begin{columns}
    \column{0.6\textwidth}
      \begin{itemize}
        \item A future possibility for the post-LHC era at CERN \vspace{0.2cm}
        \item 3 options of circular colliders \vspace{0.2cm}
          \begin{itemize}
            \item FCC-ee: electron - positron collisions
            \item FCC-hh: proton - proton collisions
            \item FCC-eh: electron - proton collisions
          \end{itemize}
        \item $\sim$100~km tunnel in Geneva area \vspace{0.2cm}
        \item FCC-ee collider parameters: \vspace{0.2cm}
      \end{itemize}
        \centering
      	\begin{tabular}{| l | c | c | c | c |}
        	\toprule
      	   Stages & Z & WW & H (ZH) & t\={t} \\
      	   \midrule
           Center of mass energy $\sqrt{s}$ [GeV] & 91.2 & 160 & 240 & 365 \\
           Average bunch spacing [ns] & 19.6 & 163 & 994 & 3396\\
      	   \bottomrule
      	\end{tabular}

    \column{0.4\textwidth}
      \centering
      \includegraphics[width=0.9\textwidth]{Figures/cernFCC}
  \end{columns}

  \end{tcolorbox}
\end{textblock}

% %%%%%%%%%%%%%%%%%%%%%%%%%%%%%%%%%%%%%
% %%% Block %%%
% %%%%%%%%%%%%%%%%%%%%%%%%%%%%%%%%%%%%%
\begin{textblock}{38}(45.5, 18)
  \begin{tcolorbox}[title=FCCSW: simulation software for FCC]

  \vspace{0.5cm}
  \begin{itemize}
    \item Common \textsc{Geant4}-based software for all FCC experiments (ee, hh \& eh)~\cite{FCCSW} \vspace{0.5cm}
    \item Detector and physics studies \vspace{0.5cm}
      \begin{itemize}
        \item Fast \& full simulations
        \item One software stack from event generation to physics analysis \vspace{0.5cm}
      \end{itemize}
    \item Collaborative approach with other CERN experiments \vspace{0.5cm}
      \begin{itemize}
        \item Gaudi from LHC~\cite{Gaudi} $\Rightarrow$ software architecture
        \item DD4hep~\cite{DD4hep} from CLIC \& LHCb $\Rightarrow$ detector description
        \item New solutions where needed
      \end{itemize}
    \item The simulation pipeline
  \end{itemize}

  \vspace{0.5cm}

  \centering
   % \scalebox{2}{
  	 \smartdiagramset{back arrow disabled=true, module minimum width=6cm, text width=6cm, module minimum height=3cm, module x sep=8cm}
  	 	\smartdiagram[flow diagram:horizontal]
  	 	{%
  	   	{Geometry\\DDhep}, Segmentation, {\textsc{Geant4} \\simulation}, Digitization%
  	 	}
     % }
     \vspace{0.5cm}
  \end{tcolorbox}
\end{textblock}

% %%%%%%%%%%%%%%%%%%%%%%%%%%%%%%%%%%%%%
% %%%                             Block                                %%%
% %%%%%%%%%%%%%%%%%%%%%%%%%%%%%%%%%%%%%
\begin{textblock}{83}(0.5, 38)
  \begin{tcolorbox}[title=The IDEA detector concept for FCC-ee]

  \begin{columns}
    \column{0.3\textwidth}
      \begin{itemize}
        \item The IDEA detector is one of the two detector concepts for the FCC-ee \vspace{0.5cm}
        \item Ultimate goal for the IDEA concept \vspace{0.5cm}
          \begin{itemize}
            \item Vertex detector: MAPS \vspace{0.2cm}
            \item Ultra-light drift chamber with particle identification \vspace{0.2cm}
            \item Double readout calorimetry \vspace{0.2cm}
            \item Aditional silicon disk layers placed in the space between the drift chamber and the dual readout calorimeter to increase the forward coverage \vspace{0.2cm}
            \item 2~T solenoidal magnetic field \vspace{0.2cm}
            \item Instrumented return yoke \vspace{0.2cm}
            \item Large tracking volume (R $\sim$ 8~m) for very weakly coupled (long-lived) particles
          \end{itemize}
      \end{itemize}


    \column{0.25\textwidth}
      \centering
      \includegraphics[width=\textwidth]{../figures/FCCeeIDEAConcept}

    \column{0.45\textwidth}

    \begin{itemize}
      \item The IDEA detector as currently simulated with FCCSW
    \end{itemize}

    \centering
    \begin{tikzpicture}
      \node[anchor=south west,inner sep=0] (image) at
      (0,0){\includegraphics[width=0.9\textwidth]{Figures/FCCeeIDEA_IR}};
      \begin{scope}[x={(image.south east)},y={(image.north west)}]
      \node[left] at (0.6, 0.7) {\textbf{Drift Chamber}};

      \draw[->, thick](0.05, 0.25) -- (0.05, 0.1);
      \node[above] at (0.05, 0.25) {\textbf{Beam Pipe}};

      \draw[->, thick](0.15, 0.45) -- (0.15, 0.22);
      \node[above] at (0.15, 0.45) {\textbf{Solenoid Shielding}};

      \draw[->, thick](0.2, 0.7) -- (0.42, 0.4);
      \node[above] at (0.2, 0.7) {\textbf{Tungsten Shielding}};

      \draw[->, thick](0.55, 0.05) -- (0.48, 0.4);
      \node[below] at (0.55, 0.05) {\textbf{Luminosity Calorimeter}};

      \draw[->, thick](0.75, 0.2) -- (0.58, 0.5);
      \node[below] at (0.75, 0.2) {\textbf{Vertex Detector}};

    \end{scope}
    \end{tikzpicture}
  \end{columns}


  \end{tcolorbox}
\end{textblock}

%%%%%%%%%%%%%%%%%%%%%%%%%%%%%%%%%%%%%
%%% Block %%%
%%%%%%%%%%%%%%%%%%%%%%%%%%%%%%%%%%%%%
\begin{textblock}{43}(0.5, 60.3)
  \begin{tcolorbox}[title=The drift chamber]

    \begin{itemize}
      \item The gas volume is divided into a set of hyperboloid layers.
      \item Each layer contains one sensitive wire for signal acquisition.
      \item Field wires surround the sensitive wires to provide homogeneous electric field for each cell.
      \item The wires are rotated with a stereo angle of 0.1~radians to improve the longitudinal resolution along them.
    \end{itemize}

    \begin{columns}
    \column{0.6\textwidth}
      \begin{itemize}
        \item The parameters of the drift chamber
      \end{itemize}

      \centering
      \begin{adjustbox}{max width=0.8\textwidth}
        \begin{tabular}{l l}
          \toprule
            Gas & $90~\%$ Helium \&\\
            & $10~\%$ isobutane ($\text{C}_{4}\text{H}_{10}$) \\
            Length & 4500~mm \\
            Inner radius & 345~mm \\
            Outer radius & 2000~mm\\
            Nb. layer & 112 \\
            Cell size & 12~mm - 14.7~mm \\
            Number of sensitive wires & 56448 \\
            Single cell resolution & 0.1~mm \\
            Longitudinal resolution & 1~mm \\
          \bottomrule
        \end{tabular}
      \end{adjustbox}

    \column{0.4\textwidth}
      \centering
      \includegraphics[width=\textwidth]{Figures/Field_sensWires.png}

    \end{columns}

  \vspace{0.5cm}

  \end{tcolorbox}
\end{textblock}


%%%%%%%%%%%%%%%%%%%%%%%%%%%%%%%%%%%%%
%%% Block %%%
%%%%%%%%%%%%%%%%%%%%%%%%%%%%%%%%%%%%%
\begin{textblock}{39.5}(44, 60.3)
  \begin{tcolorbox}[title=The simulation of the drift chamber with FCCSW]

    \begin{columns}
    \column{0.5\textwidth}
      \begin{itemize}
        \item The sensitive wires as simulated in the first layer of the drift chamber with FCCSW. \vspace{0.5cm}
        \item The DD4hep segmentation (\textsc{DDSegmentation}) is responsible to associate a hit to the wire it drifts to \vspace{0.2cm}
          \begin{itemize}
            \item Reduces the running time by avoiding to place each wire individually
          \end{itemize}
      \end{itemize}

      \column{0.5\textwidth}
        \centering
        \includegraphics[width=\textwidth]{Figures/allHits}

    \end{columns}


    \begin{columns}
      \column{0.5\textwidth}
      \begin{itemize}
        \item The coverage of the drift chamber as a function of the polar angle $\theta$ is investigated using FCCSW.
        \item High coverage in the barrel region by $\sim 112$ wires in average.
        \item In the forward region, silicon disks are foresean to increase the number of layers measuring the tracks.
      \end{itemize}

      \column{0.5\textwidth}
        \centering
        \includegraphics[width=\textwidth]{Figures/numWires}
    \end{columns}

  \end{tcolorbox}
\end{textblock}



%%%%%%%%%%%%%%%%%%%%%%%%%%%%%%%%%%%%%
%%% Block %%%
%%%%%%%%%%%%%%%%%%%%%%%%%%%%%%%%%%%%%
\begin{textblock}{43}(0.5, 85)
  \begin{tcolorbox}[title=Beam-induced backgrounds and the impact on the drift chamber]

  \begin{itemize}
    \item Three main sources of beam-induced backgrounds at FCC-ee \vspace{0.5cm}
    \begin{itemize}
      \item \textbf{Incoherent $e^+e^-$ pairs} due to bremstrahlung photons $\Rightarrow$ highest source of background \vspace{0.2cm}
      \item \textbf{$\gamma\gamma\rightarrow$ hadrons} $\Rightarrow$ Expected to have a very low impact \vspace{0.2cm}
      \item \textbf{Synchrotron radiation (SR)} $\Rightarrow$ Dictates the design of the interaction region (IR) \vspace{0.2cm}
        \begin{itemize}
          \item Defines the beampipe radius, the design of the shielding (in Tungesten)
          \item Mostly stopped by the shielding, few SR photons can hit the detector
        \end{itemize}
    \end{itemize}
  \end{itemize}

  \begin{columns}
    \column{0.5\textwidth}
      \begin{itemize}
        \item The trajectory of the $e^+e^−$ pairs in a 2~T magnetic field (using helix extrapolation).
      \end{itemize}
      \centering
      \includegraphics[width=\textwidth]{../figures/pairs_R_Z}

    \column{0.5\textwidth}

    \begin{itemize}
      \item Simulation of the hits produced in the drift chamber due to incoherent $e^+e^-$ pairs (using FCCSW)
    \end{itemize}
    \centering
    \includegraphics[width=\textwidth]{Figures/incoherent_top_Z.pdf}




  \end{columns}

  \end{tcolorbox}
\end{textblock}

% %%%%%%%%%%%%%%%%%%%%%%%%%%%%%%%%%%%%%
% %%% Block %%%
% %%%%%%%%%%%%%%%%%%%%%%%%%%%%%%%%%%%%%
 \begin{textblock}{39.5}(44, 93)
   \begin{tcolorbox}[title=Conclusions]

   \begin{itemize}
     \item Summary of the occupancy of the drift chamber due to the beam-induced backgrounds \vspace{0.3cm}
   \end{itemize}
   \centering
   \begin{adjustbox}{max width=\textwidth}
     \begin{tabular}{l c c}
       \toprule
        Background & \multicolumn{2}{c}{Average occupancy} \\
         & E\textsubscript{cm} = 91.2~GeV &  E\textsubscript{cm} = 365~GeV \\
        \midrule
        $e^+e^-$ pair background & 1.1\% & 2.9\% \\
        $\gamma\gamma\rightarrow$ hadrons & 0.001\% & 0.035\%  \\
        Synchrotron radiation & - & 0.2\% \\
        \bottomrule
     \end{tabular}
   \end{adjustbox}

   \vspace{0.3cm}

   \begin{itemize}
    \item The overall impact remains low and the results are promising for the track reconstruction with this detector.
   \end{itemize}

   \vspace{0.3cm}

  \end{tcolorbox}
 \end{textblock}

% %%%%%%%%%%%%%%%%%%%%%%%%%%%%%%%%%%%%%
% %%% Block %%%
% %%%%%%%%%%%%%%%%%%%%%%%%%%%%%%%%%%%%%
 \begin{textblock}{39.5}(44, 108.5)
   \begin{tcolorbox}[title=References]

   \printbibliography

  \end{tcolorbox}
 \end{textblock}

%--------------------------------------------%
\end{frame}
\end{document}
