\documentclass[final,xcolor={dvipsnames,svgnames,x11names,table}]{beamer}
\usetheme{RJH}
%\usetheme{Boadilla}
%\usepackage[orientation=portrait,size=a1,scale=1.3]{beamerposter}
\usepackage[orientation=portrait, size=custom, width=70, height=90, scale=1]{beamerposter}
\usepackage[absolute,overlay]{textpos}
\usepackage{pifont}
\usepackage{ulem}
\usepackage{bm}
\usepackage{siunitx}
\usepackage[export]{adjustbox}
\usepackage{gmp}
%\usepackage{enumitem}

%\DeclareGraphicsRule{.1}{mps}{*}{}

\usepackage[listings,theorems]{tcolorbox}


\usepackage{libertine}
\setlength{\TPHorizModule}{1cm}
\setlength{\TPVertModule}{1cm}

\usepackage{tikz,tikz-3dplot} %nalipour
\usetikzlibrary{shapes,arrows, decorations.pathreplacing}%, snakes} %nalipour

\def\Put(#1,#2)#3{\leavevmode\makebox(0,0){\put(#1,#2){#3}}}

% Raised Rule Command:
%  Arg 1 (Optional) - How high to raise the rule
%  Arg 2            - Thickness of the rule
\newcommand{\raisedrule}[2][0em]{\leaders\hbox{\rule[#1]{1pt}{#2}}\hspace{13.5cm}}

\newcommand{\dotrule}[1]{%
   \parbox[]{#1}{\dotfill}}

\setbeamertemplate{bibliography entry title}{}
\setbeamertemplate{bibliography entry location}{}
\setbeamertemplate{bibliography entry note}{}
\footer{}


% \usebackgroundtemplate{%%\includegraphics[width=\paperwidth]{Figures/CLIC_canvas_accelerator_small.jpg}}


\title{\Huge{CLIC vertex detector R\&D}}
\author{\vspace*{1.5cm}{\Large{Niloufar Alipour Tehrani (CERN \& ETH Z\"{u}rich)\\On behalf of the CLICdp collaboration}\\\vspace*{1cm}{\Large{13th Pisa Meeting on Advanced Detectors, 24-30 May 2015}}}}
% \footer{}
\institute{CERN}
\date{}
%\footimage{}

% \usepackage[backend=bibtex,style=numeric-comp,firstinits=true]{biblatex}
% \bibliography{bib}
% \setbeamertemplate{bibliography item}[text]
% \renewcommand*{\bibfont}{\footnotesize}

% tcolorbox styles
\tcbset{%
    noparskip,
    colback=white, %background color of the box
    colframe=i6colorblockbg, %color of frame and title background
    coltext=black, %color of body text
    coltitle=white, %color of title text
    fonttitle=\bfseries,
    alerted/.style={coltitle=red,
                     colframe=gray!40},
    subtcolorbox/.style={coltitle=black,
                     colframe=i6colorscheme3,
                     colback=white,
                     coltitle=i6colorblockbg},
    }


\begin{document}
\begin{frame}

\begin{textblock}{2}(0.8,0.25)
%\includegraphics[width=4.\textwidth]{Figures/logo_cern.pdf}
\end{textblock}
\begin{textblock}{6}(61,0.01)
%\includegraphics[width=1.5\textwidth]{Figures/logo_clic.pdf}
\end{textblock}
\begin{textblock}{20}(4.5, 11)
\textcolor{red}{\small{\url{http://clicdp.web.cern.ch/}}}
\end{textblock}


%%%%%%%%%%%%%%%%%%%%%%%%%%%%%%%%%%%%%
%%%                             Block                                %%%
%%%%%%%%%%%%%%%%%%%%%%%%%%%%%%%%%%%%%
\begin{textblock}{16.5}(0.5, 14)
  \begin{tcolorbox}[title=CLIC: the Compact Linear Collider]

    % \begin{itemize}
    % \item Concept for a future $e^{+}e^{-}$ linear collider
    % \item Staged construction \& operation:
    %   \begin{enumerate}
    %   \item $\sqrt{s}$=\SI{350}{\giga\electronvolt}: \textcolor{blue}{Higgs, top physics inc. threshold}
    %   \item $\sqrt{s}$=\SI{1.4}{\tera\electronvolt}: \textcolor{blue}{Higher precision Higgs, top Yukawa coupling, first BSM searches}
    %   \item $\sqrt{s}$=\SI{3}{\tera\electronvolt}: \textcolor{blue}{double Higgs production, high sensitivity direct and indirect BSM}
    %   \end{enumerate}
    % \item Instantaneous luminosity at \SI{3}{\tera\electronvolt}:
    %   \begin{itemize}
    %   \item $\mathcal{L}=$\SI{6e34}{\per\square\centi\meter\per\second}
    %   \end{itemize}
    % \item A possible realisation close to CERN:
    %   \begin{itemize}
    %   \item Maximum length: $\sim$\SI{50}{\kilo\meter}
    %   \end{itemize}
    % \end{itemize}

    %\vspace{0.2cm}
    %\centering
    %\includegraphics[width=0.9\textwidth]{Figures/clic_at_cern.png}

  \end{tcolorbox}
\end{textblock}
%
% %%%%%%%%%%%%%%%%%%%%%%%%%%%%%%%%%%%%%
% %%%                             Block                                %%%
% %%%%%%%%%%%%%%%%%%%%%%%%%%%%%%%%%%%%%
% \begin{textblock}{52}(17.5, 14)
%   \begin{tcolorbox}[title=CLIC detector concept]
%
%
%     \begin{columns}
%       \column{0.6\textwidth}
%       %\vspace{-0.2cm}
%       Beam structure allows for \textcolor{blue}{triggerless} readout and \textcolor{blue}{power pulsing} of the detectors:
%       \begin{columns}
%         \column{0.4\textwidth}
%
%           \begin{itemize}
%           \item 312 bunches per train of \SI{156}{\nano\second}
%           % \item Bunch crossings every \SI{0.5}{\nano\second}
%           % \item Train duration: \SI{156}{\nano\second}
%           %\item 312 bunches per train
%           \item Train repetition: \SI{20}{\milli\second}
%           \end{itemize}
%
%         \column{0.6\textwidth}
%         \centering
%         %\includegraphics[width=0.6\textwidth]{Figures/CLICbeam.png}
%       \end{columns}
%
%       \vspace{0.8cm}
%       \begin{tcolorbox}[subtcolorbox, title=Vertex detector requirements]
%         \begin{columns}
%           \column{0.7\textwidth}
%           \vspace{-0.5cm}
%           \begin{itemize}
%           \item \textcolor{blue}{Aim}: efficient identification of heavy quarks in high occupancy.
%           \item Multi-layer barrel and endcap pixel detectors.
%           \item \textcolor{blue}{Goal for the pixel detectors}: achieve a single point resolution of $\sim$\SI{3}{\micro\meter} with \SI{25}{\micro\meter}  pixel pitch \& analog readout.
%           \item Time slicing of $\sim$ \SI{10}{\nano\second} allows to reduce the impact of beam-induced backgrounds. \vspace{0.1cm}
%           \item Material budget of $<0.2\%X_{0}$ per layer implies:
%             \begin{itemize}
%             \item \textcolor{blue}{\SI{50}{\micro\meter}} sensor on \textcolor{blue}{\SI{50}{\micro\meter}} ASIC.
%             \item Limit the power dissipation to \SI{50}{\milli\watt\per\square\centi\meter} in sensor area: \\
%               $\Rightarrow$ power pulsing \\
%               $\Rightarrow$ air-flow cooling: spiral arrangement of the modules in the vertex endcap regions
%             \end{itemize}
%           \end{itemize}
%
%           \column{0.3\textwidth}
%           \centering
%           %\includegraphics[width=0.8\textwidth]{Figures/secondary_vertex.png} \\
%           \vspace{-0.2cm}
%           %\includegraphics[width=0.7\textwidth]{Figures/single_spiral.png}
%         \end{columns}
%
%         % \begin{itemize}
%         % \item \textcolor{blue}{Goal for the pixel detectors}: achieve a single point resolution of $\sim$\SI{3}{\micro\meter} with \SI{25}{\micro\meter}  pixel pitch \& analog readout.
%         % \end{itemize}
%
%
% %         \begin{columns}
% %           \column{0.7\textwidth}
% %           \vspace{-2cm}
% %           \begin{itemize}
% %           \item Material budget of $<0.2\%X_{0}$ per layer implies:
% %             \begin{itemize}
% %             \item \textcolor{blue}{\SI{50}{\micro\meter}} sensor on \textcolor{blue}{\SI{50}{\micro\meter}} ASIC.
% %             \item Limit the power dissipation to \SI{50}{\milli\watt\per\square\centi\meter} in sensor area: \\
% %               $\Rightarrow$ power pulsing \\
% %               $\Rightarrow$ air-flow cooling: spiral arrangement of the modules in the vertex endcap regions
% %             \end{itemize}
% %           \end{itemize}
%
% %           \column{0.3\textwidth}
% % %          \vspace{10cm}
% %           \centering
% % %          \hspace{-2cm}
% %           %\includegraphics[width=0.7\textwidth]{Figures/single_spiral.png}
% %         \end{columns}
%       \end{tcolorbox}
%
%       \column{0.4\textwidth}
%       \centering
%       %\includegraphics[width=\textwidth]{Figures/CLIC_DETECTOR}
%     \end{columns}
%
%   \end{tcolorbox}
% \end{textblock}
% %%%%%%%%%%%%%%%%%%%%%%%%%%%%%%%%%%%%%
% %%%                             Block                                %%%
% %%%%%%%%%%%%%%%%%%%%%%%%%%%%%%%%%%%%%
% % \begin{textblock}{36.8}(22, 27)
% %   \begin{tcolorbox}[title=Vertex detector requirements]
% %   % \begin{block}{Vertex detector requirements}
% %     \vspace{-0.3cm}
% %     \begin{columns}
% %       \column{0.4\textwidth}
% %       \centering
% %       %\includegraphics[width=\textwidth]{Figures/VertexDetector_canvas.jpg}
%
% %       \column{0.6\textwidth}
% %       \begin{itemize}
% %       \item \textcolor{blue}{Aim}: efficient identification of heavy quarks in high occupancy.
% %       \item \textcolor{orange}{Multi-layer} barrel and endcap pixel detectors.
% %       \item \textcolor{blue}{Goal}: achieve single point resolution of $\sim$\SI{3}{\micro\meter} with \SI{25}{\micro\meter}  pixel pitch \& analog readout.
% %       \end{itemize}
% %     \end{columns}
%
% %     \begin{itemize}
% %     \item Material budget of $<0.2\%X_{0}$ per layer implies:
% %       \begin{itemize}
% %       \item \textcolor{blue}{\SI{50}{\micro\meter}} sensor on \textcolor{blue}{\SI{50}{\micro\meter}} ASIC.
% %       \item Limit the power dissipation to \SI{50}{\milli\watt\per\square\centi\meter} in sensor area $\Rightarrow$ airflow cooling \& power pulsing.
% %       \end{itemize}
% %     \end{itemize}
%
% % %  \end{block}
% %   \end{tcolorbox}
% % \end{textblock}
%
%
%
% %%%%%%%%%%%%%%%%%%%%%%%%%%%%%%%%%%%%%
% %%%                             Block                                %%%
% %%%%%%%%%%%%%%%%%%%%%%%%%%%%%%%%%%%%%
% \begin{textblock}{69}(0.5, 36.5)
%   \begin{tcolorbox}[title=R\&D on sensor and readout]
% %  \begin{block}{R\&D on sensor and readout}
%
%     \begin{columns}
%       \column{0.5\textwidth}
%       \begin{tcolorbox}[subtcolorbox, title=Test-beam campaigns]
%         \begin{columns}
%           \column{0.6\textwidth}
%           \begin{itemize}
%           \item Data recorded using the EUDET/AIDA telescope at:
%             \begin{itemize}
%             \item DESY II: \SI{5.6}{\giga\electronvolt} electron beam
%             \item CERN PS: \SI{10}{\giga\electronvolt} mixed beam
%             \item CERN SPS: \SI{120}{\giga\electronvolt} pion beam
%             \end{itemize}
%           \item The telescope contains 6 planes of Mimosa26 pixel sensors with a tracking resolution of $\sim$\SI{3}{\micro\meter} for \SI{5.6}{\giga\electronvolt} electron beam. \vspace{0.2cm}
%           \end{itemize}
%
%           \column{0.4\textwidth}
%           \centering
%           \begin{tikzpicture}
%             \node[anchor=south west,inner sep=0] (image) at (0, 0){%\includegraphics[width=\textwidth]{Figures/telescope.png}};
%             \draw[->,line width=4pt, color=red](6.5, 1.5) -- (6.5, 2.5);
%             \node[below, color=red] at (6.5, 1.5) {\textbf{DUT}};
%           \end{tikzpicture}
%         \end{columns}
%       \end{tcolorbox}
%
%       \column{0.5\textwidth}
%       \begin{tcolorbox}[subtcolorbox, title=CLICpix readout chip demonstrator]
%         \begin{columns}
%           \column{0.6\textwidth}
%
%           \begin{itemize}
%           \item ASIC in \SI{65}{\nano\meter} CMOS technology.
%           \item Matrix of $64 \times 64$ pixels, \SI{25}{\micro\meter} pixel pitch.
%           \item Simultaneous measurement of time of arrival (TOA) and time over threshold (TOT) per pixel.
%           \item Compatible with power pulsing scheme.
%           \item Selectable compression logic.
%           \end{itemize}
%
%           % \begin{itemize}
%           % \item ASIC in \SI{65}{\nano\meter} technology
%           % \item $64 \times 64$ pixels, \SI{25}{\micro\meter} pitch
%           % \item 4-bit TOA, 4-bit TOT %  $\Rightarrow$ \SI{10}{\nano\second} time stamping
%           % \end{itemize}
%
%           \column{0.4\textwidth}
%           \centering
%           %\includegraphics[width=0.75\textwidth]{Figures/CLICpixDemonstrator.png}
%         \end{columns}
%
%         % \begin{itemize}
%         % \item Compatible with power pulsing scheme.
%         % \item Selectable compression logic.
%         % \end{itemize}
%
%       \end{tcolorbox}
%
%     \end{columns}
%
%     %%%%%%%%%%%%%%%%%%%%%%%%%%%%%%%%%%%%%%%%
%     %%%% Sensors %%%%
%     %%%%%%%%%%%%%%%%%%%%%%%%%%%%%%%%%%%%%%%%
%
%     \begin{columns}
%       \column{0.5\textwidth}
%       \begin{tcolorbox}[subtcolorbox, title=Planar sensors]
%
%         \begin{columns}
%           \column{0.7\textwidth}
%           \begin{itemize}
%           \item The feasibility of thin sensors is studied using the Timepix ASIC with \SI{55}{\micro\meter} pixel pitch. \vspace{0.2cm}
%           \item \SI{50}{\micro\meter} to \SI{500}{\micro\meter} thick sensors are bump-bonded to \SI{100}{\micro\meter} to \SI{750}{\micro\meter}  thick Timepix ASICs.
%           \item Overall detection efficiency $>99\%$.
%           \end{itemize}
%
%           \column{0.3\textwidth}
%           \begin{tikzpicture}
%             \centering
%             \node[anchor=south west,inner sep=0] (image) at (0, 0){%\includegraphics[width=0.9\textwidth]{Figures/Advacam-50um-assembly.jpeg}};
%             \node[color=blue] at (3, 6) {\textbf{\small{\SI{50}{\micro\meter} sensor on}}};
%             \node[color=blue] at (3.5, 5.3) {\textbf{\small{\SI{700}{\micro\meter} Timepix ASIC}}};
%           \end{tikzpicture}
%         \end{columns}
%
%         \begin{itemize}
%         \item Charge sharing and hit resolution depend on sensor thickness:
%           \begin{itemize}
%           \item $\sim$\SI{4}{\micro\meter} resolution achievable for 2-hit clusters (including the tracking resolution).
%           \item For single-hit clusters, the resolution is determined by the pixel size.
%           \end{itemize}
%         \end{itemize}
%
%         \vspace{0.3cm}
%
%         \begin{columns}
%           \column{0.5\textwidth}
%           \centering
%           %\includegraphics[width=0.9\textwidth]{Figures/cluster_size_vs_thickness_avg.pdf}
%
%           \column{0.5\textwidth}
%           \centering
%           \begin{tikzpicture}
%             \centering
%             \node[anchor=south west,inner sep=0] (image) at (0, 0){%\includegraphics[width=0.9\textwidth]{Figures/resExampleRun1190.pdf}};
%             %% \draw[->,line width=.4pt, color=blue](4.5, 6.) -- (6, 6.);
%             \node[left, color=blue] at (7, 9.5) {\textbf{\small{Multi-hit}}};
%             \node[left, color=blue] at (7, 8.8) {\textbf{\small{clusters}}};
%
%             %% \draw[<-,line width=.4pt, color=blue](6.5, 4) -- (7.5, 4);
%             \node[right, color=blue] at (9.5, 4) {\textbf{\small{Single-hit}}};
%             \node[right, color=blue] at (9.5, 3.4) {\textbf{\small{clusters}}};
%           \end{tikzpicture}
%         \end{columns}
%
%         % \begin{columns}
%         %   \column{0.5\textwidth}
%         %   \begin{itemize}
%         %   \item Excellent detection efficiency
%         %   \end{itemize}
%
%         %   \column{0.5\textwidth}
%         %   \centering
%         %   \begin{tikzpicture}
%         %     \centering
%         %     \node[anchor=south west,inner sep=0] (image) at (0, 0){%\includegraphics[width=\textwidth]{Figures/C06-W0126_EfficiencyVsThreshold.pdf}};
%         %     \node[color=blue] at (5.1, 4) {\tiny{Nominal}};
%         %     \node[color=blue] at (4.3, 2) {\tiny{p-in-n sensor}};
%         %   \end{tikzpicture}
%
%         % \end{columns}
%
%       \end{tcolorbox}
%
%       \column{0.5\textwidth}
%       \begin{tcolorbox}[subtcolorbox, title=Active HV-CMOS sensors]
%     %     \centering
%     %     % %\includegraphics[width=\textwidth]{Figures/HVCMOS_clicpix.pdf}
%
%         \begin{columns}
%           \column{0.6\textwidth}
%           \begin{itemize}
%           \item Capacitively coupled pixel detector (CCPDv3) is used as active sensor \textcolor{blue}{$\Rightarrow$ integrates sensor and amplifier}.% glued to CLICpix ASIC.
%           \item Two-stage amplifier in each pixel.
%           \item Through a layer of glue, the CCPDv3 chip is capacitively coupled from its amplifier output to the CLICpix readout ASIC \textcolor{blue}{$\Rightarrow$ no bump-bonding.}
%           % \item CCPDv3 is capacitively coupled to CLICpix bond pads through a layer of glue
%           \end{itemize}
%
%           \column{0.4\textwidth}
%           \begin{tikzpicture}
%             \centering
%             \node[anchor=south west,inner sep=0] (image) at (0, 0){%\includegraphics[width=\textwidth]{Figures/CCPDv3_CLICpix.jpg}};
%             \node[color=blue] at (8, 2.5) {\textbf{\small{CCPDv3}}};
%             \draw[->,line width=2pt, color=blue](4, 3.5) -- (5.5, 3.5);
%             \node[left, color=blue] at (4, 3.5) {\textbf{\small{CLICpix}}};
%           \end{tikzpicture}
%         \end{columns}
%
%         %\vspace{0.9cm}
%         \begin{itemize}
%         \item CCPDv3 is implemented in \SI{180}{\nano\meter} HV-CMOS process and biased at \SI{60}{\volt} \textcolor{blue}{$\Rightarrow$ create a depletion layer with fast signal collection through drift.}
%         \item High single-hit detection efficiency (high threshold DAC corresponds to low threshold as the chip is operated in negative polarity):
%         \end{itemize}
%
%
%         \begin{columns}
%           \column{0.5\textwidth}
%           \centering
%           \begin{tikzpicture}
%           \node[anchor=south west,inner sep=0] (image) at (0, 0){%\includegraphics[width=0.8\textwidth]{Figures/glueing}};
%           \node[color=white] at (8, 6.2) {\small{\textbf{CCPDv3 pad}}};
%           \node[color=white] at (8, 5.5) {\small{\SI{19.88}{\micro\meter}}};
%           \draw[<->,line width=3pt, color=white](6, 4.8) -- (10.2, 4.8);
%
%           \node[color=white] at (8, 2) {\small{\textbf{CLICpix pad}}};
%           \node[color=white] at (8, 1.3) {\small{\SI{14.94}{\micro\meter}}};
%           \draw[<->,line width=3pt, color=white](6.5, 3) -- (9.5, 3);
%
%           \node[color=white] at (5, 4) {\small{\textbf{glue}}};
%           \draw[<->,line width=3pt, color=white](3.8, 3.5) -- (3.8, 4.5);
%           \end{tikzpicture}
%
%           \column{0.5\textwidth}
%           \begin{tikzpicture}
%             \centering
%             \node[anchor=south west,inner sep=0] (image) at (0, 0){%\includegraphics[width=0.9\textwidth]{Figures/thresholdEfficiencyNegative}};
%             \node[color=blue] at (9, 5) {\textbf{\small{CLICpix-CCPDv3}}};
%             \node[color=blue] at (9, 4.2) {\textbf{\small{work in progress}}};
%             \draw[white, fill=white] (9, 0.1) rectangle (14, 1);
%             \node[color=black] at (12, 0.5) {\textbf{\small{Threshold DAC}}};
%           \end{tikzpicture}
%         \end{columns}
%
%
%       \end{tcolorbox}
%     \end{columns}
%
%   \end{tcolorbox}
% \end{textblock}
%
% %%%%%%%%%%%%%%%%%%%%%%%%%%%%%%%%%%%%%
% %%%                             Block                                %%%
% %%%%%%%%%%%%%%%%%%%%%%%%%%%%%%%%%%%%%
% \begin{textblock}{34.2}(0.5, 72)
%   \begin{tcolorbox}[title=Power pulsing]
%
%
%     Power-delivery and power-pulsing design for low-mass vertex detector:
%
%     \vspace{0.4cm}
%     \begin{itemize}
%     \item Turn off the front-end in gaps between bunch trains to reduce average power in ASIC.
%     \item Local energy storage in Silicon capacitors and voltage regulation with low-dropout (LDO) regulators.
%     \item FPGA-controlled current source provides small continuous current.
%     \item Low-mass all-Kapton cables.
%     \item Prototype built and tested: $I_{ladder}$=\SI{300}{\milli\ampere}, P$<$\SI{45}{\milli\watt\per\square\centi\meter}.
%     \end{itemize}
%
%     \vspace{0.05cm}
%
%     \begin{columns}
%       \column{0.7\textwidth}
%       \centering
%       %\includegraphics[width=0.8\textwidth]{Figures/power_pulsing.pdf}
%
%       \column{0.3\textwidth}
%       \centering
%       %\includegraphics[width=1.\textwidth]{Figures/power_pulsing_scope.png}
%     \end{columns}
%
%   \end{tcolorbox}
% \end{textblock}
%
%
% %%%%%%%%%%%%%%%%%%%%%%%%%%%%%%%%%%%%%
% %%%                             Block                                %%%
% %%%%%%%%%%%%%%%%%%%%%%%%%%%%%%%%%%%%%
%
% \begin{textblock}{34.2}(35.2, 72)
%   \begin{tcolorbox}[title=Air-flow cooling]
%     Forced air-flow is foreseen for the heat removal of the vertex detector. %\vspace{0.5cm}
%
%     \begin{columns}
%       \column{0.5\textwidth}
%       \vspace{0.4cm}
%       \begin{itemize}
%       \item Total heat load after power-pulsing: $\sim$\SI{500}{\watt} \vspace{0.3cm}
%       \item Dry air flows through the barrel and the endcap regions.% \vspace{0.5cm}
%       \end{itemize}
%
%       \centering
%       %\includegraphics[width=0.7\textwidth]{Figures/Vertex_Flow_Streamlines.png}
%
%       \column{0.5\textwidth}
%       \vspace{-1cm}
%       \begin{itemize}
%       \item Thermal mockup built for vertex barrel and endcap regions:  %\vspace{0.5cm}
%         \begin{itemize}
%         \item Confirms the air-flow through the barrel and the endcap regions.% \vspace{0.3cm}
%         \item Temperature increase: $\sim$\SI{10}{\celsius} to \SI{35}{\celsius} \vspace{0.5cm}
%         \end{itemize}
%       \end{itemize}
%       %\includegraphics[width=0.8\textwidth]{Figures/airflowTests.jpeg}
%     \end{columns}
%
%
%   \end{tcolorbox}
% \end{textblock}
%
% %% \begin{textblock}{34.2}(35.2, 72)
% %%   \begin{tcolorbox}[title=Air-flow cooling]
%
% %%     \begin{columns}
% %%       \column{0.5\textwidth}
% %%       \begin{itemize}
% %%       \item Forced air-flow is foreseen for the heat removal of the vertex detector. %\vspace{1cm}
% %%       \item Total heat load after power-pulsing: $\sim$\SI{500}{\watt} %\vspace{0.5cm}
% %%       \item Dray air at \SI{0}{\celsius} flows through the barrel and the endcap regions.% \vspace{0.5cm}
% %%       \item Thermal mockup built vertex barrel and endcap regions:  %\vspace{0.5cm}
% %%         \begin{itemize}
% %%         \item Confirms the air-flow through the barrel and the endcap regions.% \vspace{0.3cm}
% %%         \item Temperature increase: $\sim$\SI{10}{\celsius} to \SI{35}{\celsius}% \vspace{0.6cm}
% %%         \end{itemize}
% %%       \end{itemize}
%
% %%       \column{0.5\textwidth}
% %%       \centering
% %%       %% %\includegraphics[width=0.5\textwidth]{Figures/Vertex_stream.png}\\
% %%       %\includegraphics[width=0.8\textwidth]{Figures/Vertex_Flow_Streamlines.png}\\
% %%       %\includegraphics[width=0.8\textwidth]{Figures/airflowTests.jpeg}
% %%     \end{columns}
%
%
% %%   \end{tcolorbox}
% %% \end{textblock}

%--------------------------------------------%
\end{frame}
\end{document}
