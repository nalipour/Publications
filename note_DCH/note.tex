\documentclass{cernatsnote}
\usepackage[colorinlistoftodos]{todonotes}
\usepackage{placeins}
% \usepackage[pdftex]{graphicx}
\usepackage{smartdiagram}
\usepackage{hyperref}
\usesmartdiagramlibrary{additions}
\usepackage[capitalise]{cleveref}
\usepackage{booktabs}
\usepackage{xspace,upgreek}
\usepackage{units_definitions}
\usepackage{tikz}

\usepackage{amsmath, tabularx}
\usepackage{graphicx}
\usepackage{subcaption}
\usepackage{framed}

\usepackage{listings}
\lstset{basicstyle=\ttfamily,
  showstringspaces=false,
  commentstyle=\color{red},
  keywordstyle=\color{blue}
}

\title{Simulation studies for a drift chamber at the FCC-ee experiment}
\author{
	Niloufar Alipour Tehrani \; \\
	CERN, CH-1211 Geneva, Switzerland
}
\email{niloufar.alipour.tehrani@cern.ch}
\date{\today}

\begin{document}
\maketitle

\begin{abstract}
	The physics aims at the electron-positron option for  the Future Circular Collider (FCC-ee)~\cite{Gomez-Ceballos:2013zzn}, impose high precision requirements on the vertex and tracking detectors.  The detector has also to match the experimental conditions such as the collisions rate and the presence of beam-induced backgrounds.
	A light weight tracking detector is under investigation for the IDEA (International Detector for Electron-Positron Accelerator) detector concept and consists of a drift chamber. Simulation studies of the drift chamber using the FCCSW (FCC software) are presented. Full simulations are used to study the effect of beam-induced backgrounds on this detector.
  Tracking for the drift chamber is also investigated using the Hough transformation method. A technical documentation on running the different software components is as well provided.
\end{abstract}
\\ \\ \\

\begingroup
\color{black}
\tableofcontents
\endgroup

\pagebreak

\include{simulation}
\section{Tracking}

The drift chamber, with 112 layers of wires, provides high number of measurements which can be exploited for the track reconstruction.

One of the methods we are investigated is the Hough transform as described below.

\subsection{The Hough Transform}
Initially invented for bubble chamber photographs~\cite{HTWikipedia}, the Hough Transform is a feature extraction technique used in several fields such as image analysis, computer vision and digital image procession. It allows for the identification of lines as well as other shapes such as circles or ellipses.

\subsubsection{Principle}
The simplest case of Hough transform is detecting straight lines. In the parameter space, lines are represented as a point (b, m) with \cref{lineeq}.

\begin{equation}
	y = m \cdot x + b
	\label{lineeq}
\end{equation}

\begin{figure}[ht]
	\begin{tikzpicture}[scale=1.5]
    % Draw axes
    \draw [<->,thick] (0,2) node (yaxis) [above] {$y$}
        |- (3,0) node (xaxis) [right] {$x$};
    % Draw two intersecting lines
    \draw (0,0) coordinate (a_1) -- (2, 2) coordinate (a_2);
    \draw (0,1.5) coordinate (b_1) -- (1.8,0) coordinate (b_2);

    \coordinate (c) at (intersection of a_1--a_2 and b_1--b_2);
		% right angle
    \fill[red] (c) circle (2pt);

\end{tikzpicture}
\caption{TODO: A line is representate as a point (b, m) in the parameter space according to \cref{lineeq}.}
\label{fig_lineParamSpace}
\end{figure}

With the presentation (b, m) in the parameter space, vertical lines pose problems for the unbounded slope parameter \textit{m}. The Hesse normal form as described in \cref{line_hesse} can be used as a solution to get around vertical lines, where $r$ is the shortest distance from the origin to the line and $\theta$ is the angle between the $x$ axis and the line connecting the origin with the closest point as illustrated in \cref{fig_lineParamSpace}. The (r,~$\theta$) plane is referred to as the Hough Space.


\begin{equation}
	r = x \cdot \cos(\theta) + y \cdot \sin(\theta)
	\label{line_hesse}
\end{equation}

\cref{HTLine} shows the Hough transformation applied to every point on a line. In the Hough space, all the points on the line are represented by a local maximum since they all have the same (r,~$\theta$) value.
\begin{figure}[ht]
	\centering
	\begin{subfigure}[b]{0.45\textwidth}
        \includegraphics[width=\textwidth]{figures/line.pdf}
        \caption{Parameter space}
        % \label{pointsLine}
    \end{subfigure}
		~ %
		\begin{subfigure}[b]{0.45\textwidth}
					\includegraphics[width=\textwidth]{figures/line_hough.pdf}
					\caption{Hough space}
					% \label{pointsLine2}
			\end{subfigure}
	\label{HTLine}
	\caption{A line as represented in the parameter and the Hough space.}
\end{figure}

\subsubsection{Identification of circles}

The track of a charged particle in a magnetic field follows a helicoidal trajectory. In the xy-plane, the hits follow a circular trajectory as described with \cref{circleEq} where $(a, b)$ represent the center of the circle and $R$ the radius of the circle.

\begin{equation}
  {(x-a)}^2 + {(y-b)}^2 = R^2
	\label{circleEq}
\end{equation}

The Hough transformation gets better results when applied to lines. For this reason, first the conformal mapping~\cite{Hansroul:1988wa} is first applied to map circular hits into lines using \cref{conformalTrans}.

\begin{equation}
  u = {{x} \over {x^2+y^2}} , \,\,\,\, v = {{y} \over {x^2+y^2}}
	\label{conformalTrans}
\end{equation}

The conformal mapping maps a circle to a line if and only if the circle passes from the origin or following the condition as described in \cref{conformalTrans_condition} and the straight lines are described by \cref{equation_straightLine}.

\begin{equation}
  a^2 + b^2 = R^2
	\label{conformalTrans_condition}
\end{equation}

\begin{equation}
  v = {1 \over {2b}} - u {a \over b}
	\label{equation_straightLine}
\end{equation}

If the condition in \cref{conformalTrans_condition} is not satisfied, some correction terms are needed for the transformation in \cref{conformalTrans} to transform circles into lines.

The Hough transformation is then applied to the straight lines using \cref{Eq_HT_circle}.

\begin{equation}
	\rho = u \cdot cos(\phi) + v \cdot sin(\phi)
	\label{Eq_HT_circle}
\end{equation}

The radius of the circle $R$ is connected the $\rho$ parameter of the Hough transformation by \cref{Eq_R_rho}.

\begin{equation}
	R = {1 \over {2 \cdot \rho}}
	\label{Eq_R_rho}
\end{equation}

Finally, the center of the circle is extracted from the Hough transformation using \cref{Eq_HT_centers}.


\begin{equation}
	a = {cos(\phi) \over {2 \cdot \rho}},
\,\,\,\,
	b = {sin(\phi) \over {2 \cdot \rho}}
\label{Eq_HT_centers}
\end{equation}


\begin{figure}[ht]
	\centering
	\begin{subfigure}[b]{0.3\textwidth}
        \includegraphics[width=\textwidth]{figures/circle.pdf}
        \caption{}

    \end{subfigure}
		~ %
		\begin{subfigure}[b]{0.3\textwidth}
					\includegraphics[width=\textwidth]{figures/circle_CT.pdf}
					\caption{}
			\end{subfigure}
			~ %
			\begin{subfigure}[b]{0.3\textwidth}
						\includegraphics[width=\textwidth]{figures/circle_HT.pdf}
						\caption{}
				\end{subfigure}
	\label{HTcircle}
	\caption{Circles (a) as represented after a conformal mapping (b) and after the Hough transformation (c).}
\end{figure}

The Hough transformation is a periodic function and the points in the $\rho-\theta$ plane are bounded by $\theta \in \left[0, 2\pi\right]$ and $\rho \in \left[-\sqrt{u^2+v^2}, \sqrt{u^2+v^2}\right]$. It is important to note that the points $(\theta, \rho)$ and $(\theta+\pi, -\rho)$ describe the same line. In the studies presented in this document, to remove the ambiguity, $\rho$ is limited to positive values.



\subsection{Identification of single tracks}
The detection of single particle tracks simulated with FCCSW has been investigated.

Simulations using a $2.4\,\gev$ muon particle gun have been done. In a 2~T magnetic field, the bending radius is 4~m. An event is displace in \cref{fig_pgun_3d}.

\begin{figure}[ht]
	\centering
	\includegraphics[width=0.6\textwidth]{figures/3D_pgun.pdf}%
	\caption{Simulated hits in the drift chamber for a $2.4\,\gev$ muon in a 2~T magnetic field.}
	\label{fig_pgun_3d}
\end{figure}

\cref{fig_pgun_CT} shows the conformal transformation to the (x-y) position of the simulated hits in the drift chamber and the hits are mapped to a line.

\begin{figure}[ht]
	\centering
	\includegraphics[width=0.6\textwidth]{figures/CT_pgun.pdf}%
	\caption{Conformal transformation for the hits as shown in \cref{fig_pgun_3d}.}
	\label{fig_pgun_CT}
\end{figure}

Finally the Hough transformation is applied to the conformal transform (\cref{fig_pgun_CT}) and the result is shown in \cref{fig_pgun_HT}.

\begin{figure}[ht]
	\centering
	\includegraphics[width=0.6\textwidth]{figures/HT_pgun.pdf}%
	\caption{Hough transformation applied to the conformal transform as shown in \cref{fig_pgun_CT}.}
	\label{fig_pgun_HT}
\end{figure}

\subsection{Hough transformation for the incoherent pair background}
\subsection{Hough transformation and the simulation of jets}

For a center-of-mass energy of $\sqrt{s} = 91 \,\gev$, the simulation of the jets by the decay Z~$\rightarrow$~d\={d}.

\begin{figure}[ht]
	\centering
	\begin{subfigure}[b]{0.3\textwidth}
        \includegraphics[width=\textwidth]{figures/Zdd_3D.pdf}
        \caption{}

    \end{subfigure}
		~ %
		\begin{subfigure}[b]{0.3\textwidth}
					\includegraphics[width=\textwidth]{figures/CT_Zdd.pdf}
					\caption{}
			\end{subfigure}
			~ %
			\begin{subfigure}[b]{0.3\textwidth}
						\includegraphics[width=\textwidth]{figures/HT_Zdd.pdf}
						\caption{}
				\end{subfigure}
	\label{HTZdd}
	\caption{Z $\rightarrow$ d\={d}}
\end{figure}

\section{Conclusions}

The drift chamber for the IDEA detector concept has been implemented in the FCC common software (FCCSW). The full simulation chain has been implemented and validated. The impact of beam-induced backgrounds have been studied in simulations. The most important contribution belongs to the incoherent $e^+e^-$ pair particles and it is important to study the feasibility of track reconstruction despite this background.

Since the drift chamber offers a high number of measurement layers, the Hough transformation could be a promising method for pattern recognition. The Hough transformation have been explored for the reconstruction of the tracks in the drift chamber. The parameters for this method need to be optimized. The combination of the seeding information coming from the vertex detector with the Hough transformation needs to be investigated and could provide high tracking efficiencies (even in the presence of beam-induced backgrounds).

\section{Technical documentation}

In this chapter, the a tutorial for setting up the simulation and the analysis of the IDEA detector is described. More information on the FCC software is available on the \href{http://fccsw.web.cern.ch/fccsw/}{FCCSW webpage}~\cite{FCCSW}.

\subsection{Installation of FCCSW from GitHub}

First, fork the FCCSW repository on \href{https://github.com/HEP-FCC/FCCSW}{GitHub}. Log in to \textsc{lxplus SLC6} and clone your fork of the project.

\begin{lstlisting}[language=bash,caption={Clone your fork of the FCCSW repository.}]
  git clone https://github.com/[your-github-username]/FCCSW.git
  cd FCCSW
\end{lstlisting}

Make sure to correctly set remote for the GitHub repository using the following \href{https://help.github.com/en/articles/configuring-a-remote-for-a-fork}{guide}\footnote{\href{https://help.github.com/en/articles/configuring-a-remote-for-a-fork}{https://help.github.com/en/articles/configuring-a-remote-for-a-fork}}. To sync your fork with the upstream repository use the instructions given in this \href{https://help.github.com/en/articles/syncing-a-fork}{guide}\footnote{\href{https://help.github.com/en/articles/syncing-a-fork}{https://help.github.com/en/articles/syncing-a-fork}}.

Setup the environment in order to build or use the software.
\begin{lstlisting}[language=bash,caption={Setup the environment and compile.}]
  source ./init.sh
\end{lstlisting}




\bibliography{references}
\bibliographystyle{plain}

\end{document}
