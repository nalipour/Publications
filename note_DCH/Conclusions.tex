\section{Conclusions}

The drift chamber for the IDEA detector concept has been implemented in the FCC common software (FCCSW). The full simulation chain has been implemented and validated. The impact of beam-induced backgrounds have been studied in simulations. The most important contribution belongs to the incoherent $e^+e^-$ pair particles and it is important to study the feasibility of track reconstruction despite this background.

Since the drift chamber offers a high number of measurement layers, the Hough transformation could be a promising method for pattern recognition. The Hough transformation have been explored for the reconstruction of the tracks in the drift chamber. The parameters for this method need to be optimized. The combination of the seeding information coming from the vertex detector with the Hough transformation needs to be investigated and could provide high tracking efficiencies (even in the presence of beam-induced backgrounds).
